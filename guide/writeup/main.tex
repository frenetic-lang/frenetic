\newif\ifdraft\drafttrue
\newif\ifcamera\camerafalse
\newif\iffull\fullfalse
\newif\ifcolor\colorfalse

% For per-person control of tex'ing, put commands like \fulltrue
% in a file called texdirectives.tex. 
\makeatletter \@input{texdirectives} \makeatother

\documentclass{article}

\usepackage{xcolor}
\usepackage{tikz}
\usetikzlibrary{plotmarks}

\usepackage{amsmath}
\usepackage{amssymb}
\usepackage{src2tex}
\usepackage{syntax}
\usepackage{rotating}
\usepackage{balance}
\usepackage{url}

\ifcolor
\usepackage[colorlinks=true,linkcolor={pennblue},citecolor={pennred}]{hyperref}
\else
\usepackage[colorlinks=true,linkcolor={black},citecolor={black},urlcolor={black}]{hyperref}
\fi

\definecolor{cornellred}{RGB}{179,27,27}
\definecolor{pennblue}{cmyk}{1,.65,0,.3}
\definecolor{pennred}{cmyk}{0,1,.65,.34}

\newcommand{\finish}[2][cornellred]{\ifdraft\textcolor{#1}{\textbf{[#2]}}\fi}
\newcommand{\jnf}[1]{\finish[cornellred]{#1 --JNF}}
\newcommand{\dpw}[1]{\finish[cornellred]{#1 --DPW}}
\newcommand{\arjun}[1]{\finish[cornellred]{#1 --ARJUN}}

\newcommand{\code}[1]{\texttt{#1}}

\newcommand{\eg}{\emph{e.g.}}
\newcommand{\ie}{\emph{i.e.}}
\newcommand{\etc}{\emph{etc.}}

\usepackage{local}

\title{Frenetic Tutorial}

\author{Team Frenetic}

\begin{document}

\maketitle

\section{Introduction}

The goal of this tutorial is to teach readers how to program a
Software-Defined Network (SDN) running OpenFlow using the Frenetic
programming language.  This involves explaining the syntax and
semantics of Frenetic and illustrating its use on a number of simple
examples.  Along the way, there are a number of exercises for the
reader.  Solutions appear online.

\section{Motivation}

\dpw{Plagiarized in part from IEEE overview paper.}

Traditional networks are built out of special-purpose devices running
distributed protocols that provide functionality such as routing,
traffic monitoring, load balancing, NATing and access control. These
devices have a tightly-integrated control and data plane, and network
operators must separately configure every protocol on each individual
device. This configuration task is a challenging one as network
operators must struggle with a host of different baroque, low-level,
vendor-specific configuration languages.  Moreover, the pace of
innovation is slow as device internals and APIs are often private and
proprietary, making it difficult to develop new protocols or
functionality to suit client needs.

Recent years, however, have seen growing interest in software-defined
networks (SDNs), in which a logically-centralized controller manages
the packet-processing functionality of a distributed collection of
switches. SDNs make it possible for programmers to control the
behavior of the network directly, by configuring the packet-forwarding
rules installed on each switch.  Moreover, the Open Networking
Foundation is committed to developing a standard, open, vendor-neutral
protocol for controlling collections of switches.  This protocol is
OpenFlow.

SDNs can both simplify existing applications and also serve as a
platform for developing new ones. For example, to implement
shortest-path routing, the controller can calculate the forwarding
rules for each switch by running Dijkstra’s algorithm on the graph of
the network topology instead of using a more complicated distributed
protocol. To conserve energy, the controller can selectively shut down
links or even whole switches after directing traffic along other
paths. To enforce fine-grained access control policies, the controller
can consult an external authentication server and install custom
firewall rules.

But although SDNs makes it possible to program the network, they do
not make it easy. Protocols such as OpenFlow expose an interface that
closely matches the features of the underlying switch
hardware. Roughly speaking, OpenFlow allows programmers to manually
install and uninstall individual packet-processing rules.
First-generation controller systems such as NOX, Beacon, and
Floodlight support the same low-level interface, which forces
applications to be implemented using programs that manipulate the
fine-grained state of individual devices.  Unfortunately, it is
extremely difficult to develop independent program components, such as
a router, firewall and network monitor, that collaborate to control
the flow of traffic through a network since the application must
ultimately install a \emph{single} set of low-level rules on the underlying
switches.  This single set of rules must simultaneously implement the
desired high-level semantics for each independent high-level
component.

In addition, a network is a distributed system, and all of the usual
complications arise—in particular, control messages sent to switches
are processed asynchronously. Programming asynchronous, distributed
systems is notoriously difficult and error prone.  Network programmers
require require support to get this right.

The goal of the Frenetic language is to raise the level of abstraction
for programming SDNs. To replace the low-level imperative interfaces
available today, Frenetic offers a suite of declarative abstractions
for querying network state, defining forwarding policies, and updating
policies in a consistent way.  These constructs are designed to be
\emph{modular} so that individual policies can be written in isolation, by
different developers and later composed with other components to
create sophisticated policies. This is made possible in part by the
design of the constructs themselves, and in part by the underlying
run-time system, which implements them by compiling them down to
low-level OpenFlow forwarding rules.  Our emphasis on modularity and
composition—the foundational principles behind effective design of any
complex software system—is the key feature that distinguishes Frenetic
from other SDN controllers.

\section{Programming Static Network Policies}

\dpw{I'm going to describe policies a functions producing sets of packets.
Kill me later?}

A Frenetic policy describes how a collection of switches
forwards packets from one location to another.  We call a Frenetic policy
\emph{static} when it is fixed ahead of time,
does not change, and does not depend upon the packets flows that
appear in the network.  We focus on static policies in this
section of the tutorial.

The Frenetic programming paradigm encourages users to think of static
policies as abstract \emph{functions} and to ignore how these functions
are actually implemented on switch hardware.  Our compiler will take
care of implementing the functions --- programmers need only concern
themselves with specifying the right functionality.  

More specifically, each policy is a function
from a \emph{located packet} to a set of new \emph{located packets}.
A \emph{located packet} is simply a record that contains one
field for each OpenFlow-supported packet header (\code{srcMac}, \code{dstMac},
\code{vlan}, \code{srcIP}, \code{dstIP}, \code{tcpSrcPort}, \code{tcpDstPort}, 
\code{frameType}) as well as fields specifying the packet location 
in the network (\code{switch} and \code{inPort}).

To understand how a packet flows through a network, a programmer must
analyze both the current Frenetic policy $P$ and the network topology
$T$.  The policy is a function that explains how a switch should move
a packet from an input port to an output port.  The topology is a
function that explains how a packet moves from the outport of one
switch, across a link, to the inport of some other switch.  Hence,
given a located packet $p_0$, we can trace its path through the
network by first applying the policy function $P(p_0)$, generating a
set of (possibly zero) packets $\{p_1,\ldots,p_k\}$ at outports on a
switch.  For simplicity, let's assume the result $P(p_0)$ contains
just one packet ($p_1$) (\ie, it is a normal forwarding policy, not a
broadcast).  Next, we apply the topology function $T$ to generate a
packet $p_1'$ accross the other side of the link at some new switch.
Then we apply the policy function $P$ again: $P(p_1')$ generating some
further set of results.  And apply the topology function $T$ to each
element of that result set.  In summary, one traces the flow of
packets through a network by alternately applying the policy function
$P$ and the topology function $T$.

The bottom line is that (static) Frenetic is just a domain-specific
language for writing down functions that determine how switches
forward packets.

\subsection{Introductory Examples}

\begin{progeg}
stuff
\end{progeg}

Describe the basic semantics and concepts.

- located packets
- policies are functions
- basic functions


Static NetCore Programming Examples
------------------------------------

0. Review the topology in tutorial-topo.  Define some user-friendly switch names for our topology too:

1. Write a program to route all ip traffic as follows: 
  - packets with destination ip 10.0.0.10 arriving at switch C go to host 10.  
  - packets with destination ip 10.0.0.20 arriving at switch C go to host 20.  
  - packets with destination ip 10.0.0.30 arriving at switch D go to host 30.  
  - packets with destination ip 10.0.0.40 arriving at switch D go to host 40.  
  - flood all arp packets arriving at any switch
Try out your program by pinging 10 from 20 and 20 from 10.  What happens?
What happens if you ping 10 from 30?

Start your work with this handy wrapper to handle flooding of arp packets.
Replace "drop" below with some other policy that solves the problem.

```
let arpify (P:policy) =
  if frameType = arp then all
  else P
  
let my_sol1 = arpify(drop)
```

2. Extend the program written in (a) to route all traffic between
10, 20, 30, 40.

3. Now consider the program that routes all traffic between all hosts,
which we will provide.

4. Use the NetCore query facility to discover the set of all TCP ports 
being used by either machines 10 or 20.  Here is a [list of common port
numbers](http://packetlife.net/media/library/23/common-ports.pdf)
Note: some protocols are sending a lot of traffic (eg: HTTP, on port
80).  As you investigate, narrow your searches to find the "needle in
the haystack."  There is one host sending a small amount of traffic to
a non-standard port in a nonstandard format.  Which host is it?  Print
the packets using that port.  They contain a secret message.

5. Construct a firewall for the network that enforces the following policy.
Compose it with the routing policy defined in part 3.
  - Machines 10, 20, 30, 40 are trusted machines.  
  - Machine 50 is an untrusted machine.  
  - Each of these machines is serving files using HTTP (TCP port 80).  
  - Machines 10 and 20 have private files that may not be read by untrusted machines using HTTP.  
  - Machines 30, 40, 50 have public files that may be read by any machine.
  - All traffic not explicitly prohibited must be allowed to pass through the network.



\paragraph*{Acknowledgements.}
%
This work is supported in part by the NSF under grant
CNS-1111698, the ONR under award N00014-12-1-0757, and by a Google
Research Award.

{
\bibliographystyle{abbrvnat} % We recommend abbrvnat bibliography style.
\bibliography{main}
}
\end{document}
